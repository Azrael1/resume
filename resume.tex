% resume.tex
%
% (c) 2002 Matthew Boedicker <mboedick@mboedick.org> (original author) http://mboedick.org
% (c) 2003-2007 David J. Grant <davidgrant-at-gmail.com> http://www.davidgrant.ca
%
% This work is licensed under the Creative Commons Attribution-ShareAlike 3.0 Unported License. To view a copy of this license, visit http://creativecommons.org/licenses/by-sa/3.0/ or send a letter to Creative Commons, 171 Second Street, Suite 300, Sasn Francisco, California, 94105, USA.
%To get the shading to work:
%DVI->PS->PDF
%On command prompt Try (where 'resume' is name whatever this is saved as):
%latex resume.tex
%dvips -Ppdf -t letter resume.dvi
%ps2pdf resume.ps

\documentclass[letterpaper, 11pt]{article}

%-----------------------------------------------------------
%Margin setup
\usepackage[bottom=0.4in, top=0.4in, right=0.4in, left=0.4in]{geometry}
%\setlength{\voffset}{0.1in}
\setlength{\paperwidth}{8.5in}
\setlength{\paperheight}{11in}
\setlength{\headheight}{0in}
\setlength{\headsep}{0in}
\setlength{\textheight}{11in}
\setlength{\textheight}{10in}
\setlength{\topmargin}{-0.5in}
\setlength{\textwidth}{7.5in}
\setlength{\topskip}{0in}
\setlength{\oddsidemargin}{-0.6in}
\setlength{\evensidemargin}{0in}
%-----------------------------------------------------------
%\usepackage{fullpage}
\usepackage{shading}
\usepackage{hyperref}
%\textheight=9.0in
\pagestyle{empty}
\raggedbottom
\raggedright
\setlength{\tabcolsep}{0in}

%-----------------------------------------------------------
%Custom commands
\newcommand{\resitem}[1]{\item #1 \vspace{-2pt}}
\newcommand{\resheading}[1]{{\large \parashade[.9]{sharpcorners}{\textbf{#1 \vphantom{p\^{E}}}}}}
\newcommand{\ressubheading}[4]{
\begin{tabular*}{6.5in}{l@{\extracolsep{\fill}}r}
		\textbf{#1} & #2 \\
		\textit{#3} & \textit{#4} \\
\end{tabular*}\vspace{-6pt}}

\newcommand{\ressubheadinga}[2]{
\begin{tabular*}{7.11in}{l@{\extracolsep{\fill}}r}
		\textbf{#1} & #2 \\
\end{tabular*}}

\newcommand{\ressubheadingb}[2]{
\begin{tabular*}{7.11in}{l@{\extracolsep{\fill}}r}
		\textit{#1} & \textit{#2} \\
\end{tabular*}\vspace{-6pt}}

%-----------------------------------------------------------


\begin{document}
\begin{tabular*}{7.5in}{l@{\extracolsep{\fill}}r}
\textbf{\Large Bhavishya Pohani} \vspace{1pt}\\
bhavishya.pohani@gmail.com &\href{https://www.linkedin.com/in/bhavishya-pohani-ba10b3a4/}{LinkedIn} \\
(+91) 7048656359 & \href{https://www.github.com/Azrael1}{Github}\\
\end{tabular*}
%\bigskip
\vspace{-0.15in}
%\textbf{Objective:} Not sure if needed 
\resheading{Education}
\vspace{-0.235in}
\begin{itemize}

\item
	\ressubheadinga{National Institute of Technology - Surat}{Gujarat, India}
	\ressubheadingb{Bachelor of Technology in Electronics and Communication Engineering}{August 2012 - May 2016}

\end{itemize}

\vspace{-0.16in}
\resheading{Work Experience}
\vspace{-0.235in}

\begin{itemize}
\item
	\ressubheadinga{BridgeI2I Analytics}{Bengaluru, India}
	\ressubheadingb{Senior Associate Consultant}{June 2019 - Present}
	\begin{itemize}
        \resitem{Single handedly developed a Standard Industrial Classification \emph{website-classifier}; solving a 1200-class multi-label, multi-class classification problem using language representation model \href{https://arxiv.org/abs/1810.04805}{\emph{BERT}}. Model performance at par with humans, \& improved over current in-production model by a lift of 15\%}
        \resitem{Re-purposed BERT by removing it's limitation of 512 words, added confidence calibration (\href{https://arxiv.org/pdf/1706.04599.pdf}{\emph{temperature scaling}}), added \href{https://arxiv.org/pdf/1802.05300.pdf}{\emph{gold loss correction}} to deal with noisy \& untrusted labels; used \emph{Tensor Processing Units} to scale up training process to a million records}
        \resitem{Received 'Over \& Above Award' for exemplary work under tight constraints}
	\end{itemize}
	\ressubheadingb{Associate Consultant}{Dec 2017 - May 2019}
	\begin{itemize}
        \resitem{Implemented \emph{xgboost} models to predict loan foreclosure at different time intervals to boost customer retention \& decrease foreclosure rates by 20\% by customizing loans according to risk predictions}
        \resitem{Collaborated with 2 team members to form framework for proactive detection of dealer-side fraud for a consumer durable lending business}
	\end{itemize}	
\end{itemize}

\begin{itemize}
\item
	\ressubheadinga{Arya.ai}{Mumbai, India}
	\ressubheadingb{Machine Learning Research Scientist}{December 2016 - November 2017}
	\begin{itemize}
            \resitem{Built a large scale cheque automation system wherein fields such as date, amount, etc is extracted using \emph{CNNs}. Signature is localized through \href{https://arxiv.org/abs/1506.02640}{\emph{YOLO}} like architecture. System processes 100k cheques in 4 hours \& decreases the need for human involvement by 70\%}
            \resitem{Trained \& deployed face recognition API service using \href{arxiv.org/abs/1503.03832}{\emph{FaceNet}} with accuracy 90\%}
            \resitem{Put into practice an OCR system using \emph{Tesseract} OCR which used a mix of neural networks. Added rule based systems to handle very noisy documents with an accuracy of 84\%}
            \resitem{Recruited data scientists for the organization across 4 recruitment drives, interviewed 30+ candidates}
	\end{itemize}
\end{itemize}

\begin{itemize}
\item
    \ressubheadinga{Mu Sigma}{Bengaluru, India}
    \ressubheadingb{Trainee Decision 
    Scientist}{May 2016 - November 2016}
    \begin{itemize}
        \resitem{Coded a decision tree classifier in \emph{Scala} for an internally used machine learning library}
        \resitem{Leveraged \emph{Spark} to allow the classifier to handle incremental big data in a distributed fashion}
    \end{itemize}
\end{itemize}

\vspace{-0.16in}
\resheading{Projects and Contributions}
\vspace{-0.235in}

\begin{itemize}
\item
    \ressubheadinga{(Undergraduate Project) \href{https://www.github.com/Azrael1/diabetic-retinopathy}{Diabetic Retinopathy Classification}}{\emph{September 2015 - April 2016}}
    \vspace{-0.25in}
    \begin{itemize}
            \resitem{Wrote a \emph{CNN} from scratch using \emph{Theano} that classified images of eyes in 5 levels of diabetes. Handled severe class imbalance \& noise in image data}
            \resitem{Worked under the guidance of Dr. Kishor Upla \& Dr. Mukesh Zaveri}
    \end{itemize}
\item
    \ressubheadinga{\href{https://www.github.com/Azrael1/Seq-Gen}{Text generation using Recursive \& Reccurent Nets}}{\emph{May 2015 - August 2015}}
    \vspace{-0.25in}
    \begin{itemize}
            \resitem{Developed a custom network - using a mix of Recurrent \& Recursive neural networks created to generate text, built upon on Karpathy's Char-RNN}
    \end{itemize}
\end{itemize}

\vspace{-0.16in}
\resheading{Skills}
\vspace{-0.10in}

\textbf{Languages:} Python, R, Scala, C/C++, SQL, Latex. \textbf{Libraries:} TensorFlow, OpenCV, Scikit-Learn, Pandas, NLTK, Pyspark, GCP, Theano.
\textbf{Platforms:} Apache Spark, Google Cloud Project, Azure Databricks, AWS

\end{document}
